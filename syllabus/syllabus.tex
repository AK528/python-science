\documentclass[11pt]{article}

\usepackage[margin=0.75in,tmargin=0.5in]{geometry}
\usepackage{mathpazo}

\usepackage{inconsolata}
\usepackage[T1]{fontenc}

\usepackage{hyperref}

\usepackage{fancyhdr}

\pagestyle{fancy}

\fancyhead{}
\fancyfoot[LO,LE]{Python for Scientific Computing/Spring 2015}
\fancyfoot[RO,RE]{revision 1.03 (2015-01-23)}
\fancyfoot[CO,CE]{\thepage}
\renewcommand{\headrulewidth}{0.0pt}
\renewcommand{\footrulewidth}{0.0pt}


\newenvironment{squishlist}
  {\begin{itemize}
    % set spacing between items
    \addtolength{\itemsep}{-0.33\baselineskip}
    % set spacing between lines
    %\addtolength{\baselineskip}{-0.25\baselineskip}
   }
  { \end{itemize} }


\begin{document}

\begin{center}
{\Large \bf PHY 683: Python for Scientific Computing} \\[0.25em]
{\em instructor}\/: Michael Zingale, ESS 452, michael.zingale@stonybrook.edu \\
{\em date/location}\/: Mondays, 3:00--3:53~pm in Physics P112
\end{center}

\noindent {\bf Learning Goals:} \\
The learning goal of this course is to enhance scholarship in an area
of active research, by learning how to apply python to problems
in your field.

\ \\[-2mm]
\noindent {\bf Credit:} \\
This is a 1-credit course.


\ \\[-2mm]
\noindent {\bf Contacting the Instructor:}  \\
{\em e-mail:} michael.zingale@stonybrook.edu ({add ``PHY 683'' to the
 start of the subject line of any e-mail}).
%
\begin{tabbing}
\noindent {\em office hours:} \= Tues.~~  \=1:00~pm \=to \=2:30~pm \\
                              \> Thurs.\ \>1:00~pm \>to \>2:30~pm
\end{tabbing}
Generally, I am in my office every day this semester, except for Wednesdays.

\ \\[-2mm]
\noindent {\bf Webpage:} \\
The webpage for this class is \url{http://bender.astro.sunysb.edu/classes/python-science} .

\ \\[-2mm]
\noindent {\bf Texts:} \\
There are no textbooks for this class.  Online information
will be linked to from the course webpage.

\ \\[-2mm]
\noindent {\bf Lecture Topics:} \\
We will (try to) discuss the following topics in the course:
%
\begin{squishlist}
\item {\em Basics of computing} (0.5 lectures)

\item {\em Introduction to python} (3.5 lectures) \\
  data structures and control statements, functions, classes, popular modules \\
  IPython and notebooks

\item {\em Software engineering practices} (1 lecture) \\
  including {\tt git} and github

\item {\em Introduction to the NumPy array library} (1 lecture)

\item {\em matplotlib for visualization} (1 lecture)

\item {\em SciPy and numerical methods} (3 lectures)

\item {\em Introduction to SymPy} (1 lecture)

\item {\em MayaVi for 3-d visualization} (0.5 lectures)

\item {\em f2py and C extensions} (1.5 lectures)

\item {\em Building applications / packaging} (1 lecture)
\end{squishlist}

\noindent Time-permitting, we will also discuss:
\begin{squishlist}
\item {\em GUIs} (1 lecture)

\item {\em Unit testing}

\item {\em Interacting with forms on a webpage}

\item {\em Julia}
\end{squishlist}
%
\noindent The actual course topics and time spent on each topic will depend on the
interest and the participation level of the class.

\ \\[-2mm]
\noindent {\bf Computers:} \\
%
As we all use different systems for our research, we are not meeting
in a computer lab.  Instead, you should bring your own laptop to
class.  Information on how to install python on Windows, Mac OSX, and
Linux will be posted on the class webpage. \\

\noindent
If you don't have access to a computer that can run python, see me,
and I can try to arrange something for the semester.

\ \\[-2mm]
\noindent {\bf Evaluation:} \\
%
Students are expected to attend the class and to contribute
to the discussions (by asking questions, proposing examples, or
providing demonstrations of their own).  As we meet only one hour per
week, students show plan on spending time outside of class reviewing
and practicing the material we discussed. \\

\noindent {\bf The primary place for participation is the online forum
  on the class blackboard page}.  This is the place to interact with
me and your classmates---ask anything, share examples, etc. \\

\noindent Letter grades will be based on the online participation.  A
rough guide is presented below:
\begin{itemize}
\item {\sf A\phantom{+}}: 10 postings to the discussion board
  {\bf plus} a small example showing how you can apply what
  we've discussed in class to your field.

\item {\sf A$-$}: 8 postings to the discussion board

\item {\sf B$+$}: 5 postings to the discussion board

\item {\sf B\phantom{+}}:  2 postings to the discussion board
\end{itemize}
A post does not mean a ``me to''-type post, but something either
demonstrating a problem you don't understand (giving code), asking for
some detail from the lecture to be explained, sharing a neat trick you
found, answering a classmate's question, etc.  Note that asking
questions about the content counts just as much as providing
answers---the idea is to have a discussion outside of class on the
material.

\ \\[2mm]
\noindent {\bf Required Syllabus Statements:} \\

\noindent
{\em Americans with Disabilities Act: }
%
If you have a physical, psychological, medical or learning disability
that may impact your course work, please contact Disability Support
Services, ECC (Educational Communications Center) Building, Room 128,
(631) 632-6748. They will determine with you what accommodations, if
any, are necessary and appropriate. All information and documentation
is confidential.


\ \\[-2mm]
\noindent
{\em Academic Integrity: }
%
Each student must pursue his or her academic goals honestly and be
personally accountable for all submitted work.  Representing another
person's work as your own is always wrong.  Faculty are required to
report any suspected instances of academic dishonesty to the Academic
Judiciary. Faculty in the Health Sciences Center (School of Health
Technology \& Management, Nursing, Social Welfare, Dental Medicine) and
School of Medicine are required to follow their school-specific
procedures. For more comprehensive information on academic integrity,
including categories of academic dishonesty, please refer to the
academic judiciary website at
\url{http://www.stonybrook.edu/uaa/academicjudiciary/}

\ \\[-2mm]
\noindent
{\em Critical Incident Management: }
%
Stony Brook University expects students to respect the rights,
privileges, and property of other people. Faculty are required to
report to the Office of Judicial Affairs any disruptive behavior that
interrupts their ability to teach, compromises the safety of the
learning environment, or inhibits students' ability to learn.  Faculty
in the HSC Schools and the School of Medicine are required to follow
their school-specific procedures.

\ \\[-2mm]
{\em Electronic Communication: }
%
Email to your University email account is an important way of
communicating with you for this course.  For most students the email
address is:\\[0.25em] `firstname.lastname@stonybrook.edu'\\[0.25em]
and the account can be
accessed here.
{\em It is your responsibility to read your email received at this
  account.}
For instructions about how to verify your University email address see
this: \\[0.25em]
\url{http://it.stonybrook.edu/help/kb/checking-or-changing-your-mail-forwarding-address-in-the-epo} \\[0.25em]
%
You can set up email forwarding using instructions here: \\[0.25em]
\url{http://it.stonybrook.edu/help/kb/setting-up-mail-forwarding-in-google-mail}
\\[0.25em]
%
If you choose to forward your University email to another account, we
are not responsible for any undeliverable messages.

\ \\[-2mm]
\noindent
{\em Religious Observances: }
%
See the policy statement regarding religious holidays at \\[0.25em]
{\footnotesize \url{http://www.stonybrook.edu/commcms/provost/pdfs/Policy\%20Statement\%20Regarding\%20Religious\%20Holidays\%20revised.docx}}
\\[0.25em]
%
Students are expected to notify the course professors by email of
their intention to take time out for religious observance.  This
should be done as soon as possible but definitely before the end of
the `add/drop' period.  At that time they can discuss with the
instructor(s) how they will be able to make up the work covered.






\ \\[3 mm]
\noindent
(information on the above required syllabus statements can be found at \\
\url{http://www.stonybrook.edu/commcms/provost/category/faculty/policies.html}\,)


\end{document}
